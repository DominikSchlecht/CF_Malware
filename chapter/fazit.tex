\section{Fazit}
Durch die statische Analyse ließen sich verschiedene mögliche Funktionen der Malware erschließen (siehe Sektion \ref{ref:statAnFazit}). Daraus lassen sich für Nutzer Sicherheitshinweise ableiten. So sollten diese immer zuerst die Dateiendung überprüfen, und sich nicht von Icons täuschen lassen.\\[1ex]
	
Ebenso konnte über die dynamische Analyse zumindest ein Teil der Funktionalität aufgedeckt werden (siehe Abschnitt \ref{ref:dynAnFazit}). Aus den Erkenntnissen lassen sich \textit{Indicator of Compromise} ableiten. So könnte man zum Beispiel bei dem Verdacht, dass ein Rechner mit der hier analysierten Malware infiziert ist, auf die Datei \textit{msdb} prüfen und die Autorun-Einträge untersuchen.\\[1ex]

Dass keine weiteren Aktivitäten der Malware verzeichnet werden konnten, kann sich auf mehrere Gründe zurückführen lassen. So könnte die Malware nur für eine spezielle Windows-Konfiguration entwickelt worden sein, welche hier nicht geboten wurde. Auch wäre es möglich, dass die Malware sich erst zu einem bestimmten Zeitpunkt aktiviert und bis dahin schläft. Am wahrscheinlichsten ist jedoch, dass die Malware die virtuelle Umgebung erkannt hat und daher nicht seine eigentliche Funktionalität entfaltet. Die Annahme, dass die Malware die VM erkannt hat, wird durch die Analyse von Alexey Shulmin auf \url{securelist.com}\footnote{\url{https://securelist.com/analysis/publications/69560/the-bank\\ing-trojan-emotet-detailed-analysis/}} gestützt.