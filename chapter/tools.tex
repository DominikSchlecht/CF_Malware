\section{Verwendete Tools und Infrastruktur}
In diesem Abschnitt werden die verwendeten Tools sowie die Infrastruktur dargestellt.

\subsection{Physikalisches Betriebssystem}
Als Grundsystem wurde ein \textit{Linux}-System verwendet. Dies hat den Vorteil, dass ein Großteil der sich derzeit im Umlauf befindlichen Malware für \textit{Windows} konzipiert ist und somit nur ein geringes Risiko besteht, dass sich das Grundsystem mit der Malware infiziert. Als unkompliziertes, wandelbares und trotzdem hoch modifizierbares System wurde \textit{Debian 8} gewählt. Versuche mit zum Beispiel \textit{Gentoo} zeigten Probleme mit der verwendeten Virtualisierungslösung. 

\subsection{Virtualisierungslösung}
Es gibt viele Vorteile für die Nutzung einer Virtualisierungslösung bei der Malwareanalyse, jedoch auch Nachteile. Vorteilhaft ist vor allem das Erstellen von sogenannten Snapshots, welche einen bestimmten Zustand eines Systems aufzeichnen und es möglich machen, diesen später wieder herzustellen. Zudem wird das Host-System vor der Malware geschützt. Ein Nachteil ist, dass moderne Malware immer häufiger überprüft, ob sie in einer virtuellen Umgebung ausgeführt wird. Falls ja, werden oft andere Funktionen ausgeführt, um die ursprüngliche Funktion zu verschleiern. Insgesamt überwiegen aber die Vorteile den Nachteil. Falls die Malware auf die virtuelle Umgebung prüft, muss getestet werden, ob die Prüffunktion über den Disassembler oder Debugger deaktiviert oder umgangen werden kann.

Als Virtualisierungslösung wurde \textit{VMWare Workstation 11} genutzt. Diese bietet gerade im Bereich der Netzwerk-Modifikation weitere Möglichkeiten gegenüber der kostenlosen Variante \textit{Virtualbox} von \textit{Oracle}. Die Workstation kann von der offiziellen Webseite heruntergeladen und für 30 Tage kostenlos getestet werden.

\subsection{Virtuelles Betriebssystem}
Als virtuelles Betriebssystem wurde \textit{Windows 7 Pro} verwendet. \textit{Windows 7} bietet sich an, da es derzeit eines der meist verbreiteten Betriebssysteme ist und Malware meistens für \textit{Windows} konzipiert ist.

Zudem wurde 32-bit als Architektur gewählt, um die Kompatibilität mit Tools wie \textit{Cuckoo-Sandbox} sowie mit der aktuellen Malware sicherzustellen. Außerdem wurden sowohl das \textit{UAC}, Updates sowie die Firewall deaktiviert, um der Malware eine möglichst einfache Umgebung zu bieten. Die \textit{VMWare}-Tools wurden absichtlich nicht installiert, da dies einer Malware eine sehr einfache Möglichkeit bieten würde, die Umgebung zu erkennen.

Diese Konfiguration wird als Grund-Image verwendet.

\subsection{Weitere Tools}
Neben der oben genannten Umgebung wurden zudem folgende Tools verwendet, um die Malware zu analysieren.

\subsubsection{Resource Hacker}
\label{ResourceHacker}
\textit{Resource Hacker} ist ein Tool für \textit{Windows}, welches Bestandteile einer ausführbaren Datei aufgliedert und darstellt. So können zum Beispiel schnell Icons, Menü-Elemente, Strings oder ähnliches identifiziert und extrahiert werden.\footnote{\url{http://www.angusj.com/resourcehacker/}}

\subsubsection{Dependency Walker}
\label{ref:dependencyWalker}
Der \textit{Dependency Walker} (Version 2.2)\footnote{\url{http://www.dependencywalker.com/}} ist eine freie Software, welche 32-bit und 64-bit \textit{Windows}-Anwendungen (\textit{exe}, \textit{DLLs} und weitere) analysiert und Abhängigkeiten darstellt. Für jede Abhängigkeit werden Import und Export-Funktionen angezeigt. Dies erlaubt eine grobe Einschätzung, welche Funktionalitäten das Programm bietet.

\subsubsection{PEView}
\label{ref:PEView}
\textit{PEView} zeigt Informationen über ausführbare Dateien unter Windows an (welche im \textit{PE/COFF}-Format vorliegen, was unter \textit{Windows} der Standard ist). Hier können verschiedene Informationen wie die Import- und Exporttabellen, Textsegmente oder ähnliches untersucht werden. Da \textit{PEView} diese jedoch nur sehr rudimentär aufgliedert, kann die manuelle Analyse einige Zeit dauern.

\subsubsection{RegShot}
\label{ref:ToolsRegShot}
Das Programm \textit{RegShot} erlaubt es, den derzeitigen Zustand der \textit{Windows-Registry} zu sichern und später mit einem anderen Stand zu vergleichen. Dies ist für die dynamische Analyse sehr praktisch, da man schnell die, zum Beispiel durch Malware, veränderten Einträge sehr schnell identifizieren kann. Diese geben wiederum wichtige Informationen über die Funktionsweise der Malware oder liefern \textit{Indicators of Compromise} (IOC). Eine Analyse der Registry ohne \textit{RegShot} ist möglich, wäre aber sehr aufwendig.

\subsubsection{Process Monitor}
\label{ref:ToolProcMon}
Das Tool \textit{Process Monitor} ist Teil der von \textit{Microsoft} veröffentlichten \textit{Sysinternal-Suite}\footnote{\url{https://technet.microsoft.com/en-us/sysinternals/bb842062.aspx}}. Es zeigt Zugriffe (unter anderem Festplatte, Netzwerk, Registry) aller aktuell laufenden Programme auf und erlaubt die Filterung dieser. Ebenso kann es bereits bei Systemstart mit der Protokollierung beginnen und das Verhalten von Programme aufzeichnen.

\subsubsection{Process Explorer}
\label{ref:ToolProcExp}
Das Tool \textit{Process Explorer} ist ebenfalls Teil der \textit{Sysinternal-Suite}. Es ist eine Art erweiterter \textit{Windows Task-Manager} und bietet Funktionen wie das Überprüfen der Signaturen von laufenden Prozessen oder den Upload derer Hashes zu \textit{Virustotal} (siehe \ref{ref:ToolsVirustotal}). Ebenso kann es alle von einem Prozess geladenen \textit{DLL}s anzeigen.

\subsubsection{Autoruns}
\label{ref:ToolAutoruns}
\textit{Autoruns} ist ein weiteres Tool aus der \textit{Sysinternal-Suite}. Es listet \textit{Autostart}-Einträge des Systems aus verschiedensten Quellen auf und erlaubt so zu prüfen, ob ein unerwünschtes Programm bei Systemstart ausgeführt wird. Es kann ebenfalls dazu verwendet werden, die unerwünschten Einträge zu deaktivieren.

\subsubsection{IDA Pro Free}
\label{ref:ToolIDA}
Als Disassembler wurde \textit{IDA PRO Free} (Version 5.0) verwendet. Diese Version reicht für grundlegende Analysen, jedoch sind die analysierbaren Anwendungen auf 32-bit-Anwendungen begrenzt. Da Malware jedoch so konzipiert ist, dass ein möglichst breites Spektrum an Geräten angegriffen werden kann, fällt diese Einschränkung zumeist nicht ins Gewicht. Um mit \textit{IDA PRO Free} 64-bit Malware zu analysieren, ist eine kostenpflichtige Version notwendig. Als eine kostengünstigere Alternative zu \textit{IDA PRO} kann \textit{Hopper}\footnote{\url{http://www.hopperapp.com/}} in Betracht gezogen werden. \textit{Hopper} gibt es ebenfalls in einer kostenfreien Version. Die Stärke liegt jedoch in der kostenpflichtigen Lizenz, welche ähnliche Features bietet wie \textit{IDA PRO}, jedoch wesentlich billiger und somit auch für Privatpersonen erschwinglich ist. Zudem bietet \textit{Hopper} den Vorteil, dass es vorgefertigte Versionen für \textit{OS X} und \textit{Linux} gibt. \textit{IDA PRO (Free)} liegt nur für \textit{Windows} vor, ist jedoch über \textit{Wine} relativ einfach auf \textit{Linux} installierbar.

\subsection{Webseiten}
Neben Tools auf dem Rechner können ebenfalls Webseiten bei der Analyse von Malware helfen. Folgend sind zwei dieser Seiten aufgeführt.

\subsubsection{Virustotal}
\label{ref:ToolsVirustotal}
Die Webseite \textit{Virustotal} ermöglicht die Analyse von Dateien oder URLs, jeweils im Original oder als Hash. Nach dem Upload wird die Datei/URL von einer Vielzahl von bekannten Virenscanner gescannt und die Ergebnisse zurück gegeben. Falls die Datei/URL vor nicht allzu langer Zeit bereits gescannt wurde, wird dieses Ergebnis angezeigt. Unternehmen können sich ebenso weitere Funktionen kaufen, um breitere Analysen auf der Datenbasis von \textit{Virustotal} durchzuführen.

\subsubsection{Malwr}
\label{ref:ToolsMalwr}
\textit{Malwr} basiert auf der \textit{Cuckoo}-Sandbox und stellt eine automatisierte Lösung zur dynamischen Analyse von Malware dar. Dabei werden unter anderen Screenshots erstellt, Netzwerkverbindungen aufgezeichnet oder Änderungen in der Registry festgehalten.

\subsubsection{Immunity Debugger}
\label{ref:ToolsImmunityDebugger}
Der \textit{Immunity Debugger}\footnote{\url{http://debugger.immunityinc.com/}} ist ein kostenloser Debugger für \textit{Windows}. Er zeichnet sich vor allem durch den hohen Funktionsumfang sowie um die Möglichkeit, den Debugger durch \textit{Python-Skripte} zu erweitern, aus.